% !TEX root = hw2.tex

\section{Decision-based Cascades: A Local Election [25 points]}
It's election season and two candidates, Candidate A and Candidate B, are in a
hotly contested city council race in sunny New Suburb Town. You are a strategic
advisor for Candidate A in charge of election forecasting and voter acquisition
tactics.

Based on careful modeling, you've created two possible versions of the social
graph of voters. Each graph has 10,000 nodes, where nodes are denoted by an
integer ID between 0 and 9999.  The edge lists of the graphs are provided in 
the homework bundle.
Both graphs are \textbf{undirected}.

Given the hyper-partisan political climate of New Suburb Town, most voters have
already made up their minds: 40\% know they will vote for A, 40\% know they will
vote for B, and the remaining 20\% are undecided. Each voter's support is
determined by the last digit of their node id. If the last digit is 0--3, the
node supports A. If the last digit is 4--7, the node supports B. And if the last
digit is 8 or 9, the node is undecided.

The undecided voters will go through a 10-day decision period where they choose
a candidate each day based on the majority of their friends. The \textbf{decision period}
works as follows:
\begin{enumerate}
\item The graphs are initialized with every voter's initial state ($A$, $B$, or
  undecided).
\item In each iteration, every undecided voter decides on a candidate.
  Voters are processed in increasing order of node ID.  
  For every undecided voter, if
  the majority of their friends support A, they now support A. If the majority of
  their friends support B, they now support B. ``Majority'' for A means that
  strictly more of their friends support A than the number of their friends supporting B,
  and vice versa for B (ignoring undecided friends).
\item If a voter has an equal number of friends supporting A and B, we assign support
for A or B in alternating fashion, starting with A.  
In other words, as the voters
are being processed in increasing order of node ID, the first tie leads to
support for A, the second tie leads to support for B, the third for A, the fourth for B, and so on.
This alternating assignment happens at a \textit{global level} for the whole network, across all rounds.
(Keep a single global variable that keeps track of whether the current
  alternating vote is A or B, and initialize it to A in the first round. Then as
  you iterate over nodes in order of increasing ID, whenever you assign a vote
  using this alternating variable, change its value afterwards.)
\item When processing the updates, use the values from the current iteration.
  For example, when updating the votes for node 10, you should use the updated
  votes for nodes 0--9 from the current iteration, and nodes 11 and onwards from the
  previous iteration.
\item There are 10 iterations of the process described above.
\item On the 11th day, it's election day, and the votes are counted.
\end{enumerate}

\emph{Note that only the undecided voters go through the decision process. The decision process does not change the loyalties of those voters who have already made up their minds. Voters who are initially undecided may change their mind on each iteration of this process.}

\subsection{Basic Setup and Forecasting [4 points]}
Start your work with the starter code provided.
Read in the two graphs and assign the initial vote configurations to the
network.  Then, perform the 10 iterations of the voting process.  Which candidate wins in
Graph 1, and by how many votes? Which candidate wins in Graph 2, and by how many
votes?

For sanity check, neither of these two numbers (how many votes) is larger than 300.

\subsection{TV Advertising [8 points]}
You have amassed a substantial war chest of \$9000, and you have decided to
spend this money by showing ads on the local news. Unfortunately, only 100 New
Suburb Townians watch the local news---those with ids 3000--3099.  However, your
ads are extremely persuasive, so anyone who sees the ad is immediately swayed to
vote for candidate A regardless of his/her previous decision.  You may spend
\$1000 at a time on ads.  The first \$1,000 reaches voters 3000--3009, the
second \$1000 reaches voters 3010--3019, and so on. In other words, the total of \$$k$ in advertising would reach voters with ids from 3000 to $3000 + \frac{k}{100} - 1$.  This advertising happens
before the decision period.  \emph{After voters are persuaded by your ads, they
  never change their minds again.}

Simulate the effect of advertising spending on the two possible social
graphs. First, read in the two graphs again and assign the initial
configurations as before. Now, before the decision process, you purchase \$$k$
of ads and go through the decision process of counting votes.

For each of the two social graphs, plot \$$k$ (the amount you spend) on the
x-axis (for values $k = 1000, 2000, \ldots, 9000$) and the number of votes for A minus the number of votes for B on the y-axis.  Put these on the
same plot.  What's the minimum
amount you can spend to win the election in each of the two social graphs?

\emph{Note that the TV advertising affects all of the voters who see the ads and not just those who
  are undecided.}

\subsection{Wining and Dining the High Rollers [8 points]}
\label{cascades_3}
TV advertising is only one way to spend your campaign war chest.  You have
another idea to have a very classy \$1000 per plate event for the high rollers
of New Suburb Town (the people with the highest degree in the social graph). You
invite high rollers in order of how many people they know, and everyone that
comes to your dinner is \emph{instantly persuaded to vote for candidate A
  regardless of his/her previous decision}.  
For each high roller you spend \$1000 to this persuasion.  
This event will happen before the
decision period. When there are ties between voters with the same degree, the
high roller with lowest node ID get chosen first.

Simulate the effect of the high roller dinner on the two graphs. First, read in
the graphs and assign the initial configuration as before. Now, before the
decision process, you spend \$$k$ on the fancy dinner and then go through the
decision process of counting votes.

For each of the two social graphs, plot \$$k$ (the amount you spend) on the
x-axis (for values $k = 1000, 2000, \ldots, 9000$) and the number of votes you
win by on the y-axis (that is, the number of votes for A less the number of votes for B).
What's the minimum amount you can spend to win the election in each of the two social graphs?

\emph{Note that wining and dining sways all the voters to vote for A and not
  just those who are undecided.}

\subsection{Analysis [5 points]}
Plot the degree distributions on a log-log scale of the two graphs on the same
plot (as in Question 1.1 on Problem Set 1).  Although both graphs have roughly
the same number of edges, degree distributions are actually very different.
In 1--2 sentences, briefly
summarize how this difference explains the results of 
Question \ref{cascades_3}.

\subsection*{What to submit}
\begin{enumerate}[{Page} 1:]
\setcounter{enumi}{8}
\item 
\begin{itemize}
\item Which candidate wins and by how many votes in each graph.
\end{itemize}
 
\item
\begin{itemize}
\item Plot of winning margin in each graph as a function of \$k (on the same plot)
\item The minimum amount you can spend to win the election in each graph
\end{itemize}
 
\item
\begin{itemize}
\item Plot of winning margin in each graph as a function of \$k (on the same plot)
\item The minimum amount you can spend to win the election in each graph
\end{itemize}
 
\item
\begin{itemize}
\item Log-log plot of the degree distributions (on the same plot),
\item 1--2 sentences on why the plot explains the results of Question \ref{cascades_3}.
\end{itemize}
\end{enumerate}
